\documentclass[a4paper, 12pt, titlepage]{article}

\usepackage[english]{babel}
\usepackage[T1]{fontenc}
\usepackage[utf8]{inputenc}
\usepackage[pdftex]{graphicx}
\usepackage{tabularx}
\usepackage{listings}

\lstset{ 
	language=bash,
	basicstyle=\footnotesize\ttfamily
}

\newcommand{\urequirement}[2]{
	\subsubsection{}
	\begin{tabular}{l p{10cm}}
	\bf{How to measure} & 
		#1\\
	\bf{Requirement} & 
		#2\\
	\end{tabular}
}

\newcommand{\testcase}[2]{
	\subsubsection{}
	\begin{tabular}{l p{11cm}}
	\bf{Input} & 
		#1\\
	\bf{Output} & 
		#2\\
	\end{tabular}
}

\newcommand{\ttestcase}[3]{
	\subsubsection{#1}
	\begin{tabular}{l p{11cm}}
	\bf{Input} & 
		#2\\
	\bf{Output} & 
		#3\\
	\end{tabular}
}

\newcommand{\ptestcase}[3]{
	\subsubsection{}
	\begin{tabular}{l p{10cm}}
	\bf{Prerequisites} & 
		#1\\
	\bf{Input} & 
		#2\\
	\bf{Output} & 
		#3\\
	\end{tabular}
}

\begin{document}
	
	% Framsidan
	\begin{titlepage}
		
		\includegraphics[scale=0.4]{logoNormal.png}
		
		\vspace{6cm}
		
		\begin{center}
			\Huge{\textbf{Test cases}} % Ändra dokumentets titel
			
			\vspace{0.5cm}
			
			\huge{2012-05-15} % Datum senast ändrad
		\end{center}
		
	\end{titlepage}
	
	% Innehållsförteckning
	\tableofcontents
	\newpage
	
	% Versioner (lägg till ändringar ni gör som en ny rad i tabellen nedan).
	% Under Utförda av anges LiU-id.
	\section*{Versions}
	\begin{tabularx}{1\textwidth}{|l|l|X|l|}
		\hline
		\bf{Version} & \bf{Date} & \bf{Changes} & \bf{By} \\
		\hline
		0.4 & 2012-05-15 & Corrections from feedback. & nicol271 \\
		\hline
		0.3 & 2012-05-02 & More integration tests. & nicol271 \\
		\hline
		0.2 & 2012-04-19 & Database abstraction tests. & nicol271 \\
		\hline
		0.1 & 2012-04-10 & First version. & nicol271 \\
		\hline
	\end{tabularx}
	\newpage
	
	\section*{Introduction}
	This document contains a description for test cases of the system. For information on running test scripts, please consult the \emph{Installation and operation manual}. Information absout test runs and test results may be found in the \emph{Test log} document.

	% Dokumentets innehåll nedanför här.
	\section{System quality tests}
	Tests overall, system-wide qualities.

	\subsection{Usability}
		\urequirement
		{
			A tester is given the following tasks: read a thread, reply to the thread, and create a new thread.
		}{
			3 out of 4 randomly chosen testers must be able to fullfill the tasks, see R.1.1.0.1
		}

	\subsection{Performance}
		\urequirement
		{
			By navigating through the system, measuring the response time using the net-panel in the browser web developer tools. The time is measured using a standard PC client with a broadband network connetion to a webserver.
			The response time is measured as the total time between two interactions with the application. The initial time to start the application (load the web page) is not included as reponse time.
		}{
			Response time less than 2 seconds, see R.1.2.0.1
		}

	\section{Testing the server}
	Test scripts for this tests can be found in \lstinline{test/node/server.test.js}. For information on running test scripts, please consult the \emph{Installation and operation manual}. 

	\subsection{Argument handling}
	Tests if the server handles incorrect command-line arguments.\\

		\testcase
		{
			Server is called without giving an appname to run
		}{
			A message describing the usage is shown
		}

		\testcase
		{
			An incorrect appname is given
		}{
			A message saying that the app could not be found
		}

	\subsection{App configuration}
	Tests if handling of the app-definitions is correct.\\

	This part of the testing has not been written, since the purpose of the server in the project only is to run the forum application. This kind of tests would only have been useful if the server was more generic and actually was supposed to run many different applications.


	\subsection{Networking}
	Tests if the server really can serve requests correctly.\\
	Prerequisties: The server is running an existing app which returns known data.

		\testcase
		{
			A plain HTTP GET-request without data is sent to the port that the server is listening on
		}{
			A response with HTTP-status code 200
		}

		\testcase
		{
			A plain HTTP-request is sent to the port that the server is listening on
		}{
			The body of the HTTP response consists of the known data declared in the app
		}

		\testcase
		{
			A HTTP-request with a path to a non-existing file is sent to the server
		}{
			 A response with HTTP-status code 404
		}

	\section{Testing database abstraction layer}
	Unit-testing of the part of the system which interacts directly with the database.\\

	Test scripts for this tests can be found in \lstinline{src/testdriver/tests/}. For information on running test scripts, please consult the \emph{Installation and operation manual}. 

	\subsection{Users}

		\testcase
		{
			Log into the database with a name and a callback-function
		}{
			The user is logged in and the callback-function is called
		}

		\testcase
		{
			Get list of users from database
		}{
			The logged-in users are recieved
		}

	\subsection{Categories}

		\testcase
		{
			Create a new category with a name 
		}{
			The category is created, no error-callback is recieved
			% The callback-function is called with an error
		}

		\testcase
		{
			Get categories from database
		}{
			The supplied callback-function is called with a list of all category-objects in the database
		}

	\subsection{Subforum}

		\testcase
		{
			Create a new subforum in a category
		}{
			The subforum is created, no error-callback is recieved
			% The callback-function is called with an error
		}

		\testcase
		{
			Get all subforums for a specified category
		}{
			A list of all subforum-objects for the specified category
		}

	\subsection{Threads}

		\testcase
		{
			Create a new thread in a subforum
		}{
			The thread is created, no error-callback is recieved
			% The callback-function is called with an error
		}

		\testcase
		{
			Get all threads for a specified subforum
		}{
			A list of all thread-objects for the specified subforum
		}

	\subsection{Posts}

		\testcase
		{
			Create a new post in a thread
		}{
			The post is created in the thread, no error-callback is recieved
			% The callback-function is called with an error
		}

		\testcase
		{
			Get all posts for a specified thread
		}{
			A list of all post-objects for the specified thread
		}

	\subsection{Answers}

		\testcase
		{
			Create a new answer to a thread
		}{
			The answer is created, no error-callback is recieved
			% The callback-function is called with an error
		}

		\testcase
		{
			Get all answers for a specified thread
		}{
			A list of all answer-objects for the specified thread
		}


	\subsection{Overall}

		\ttestcase{Callbacks on new data}
		{
			Gets a list of objects and then inserts a new object into the database.
		}{
			The callback-function is called twice: once with the first query result, and then again efter inserting the new object.
		}

	\section{Testing Enyo-kinds}
	Unit- and integration testing of the modules which the application consists of.

	\subsection{Utilities}

		\ttestcase{Date formatting}
		{
			Javascript date objects, such as Date(2011, 03, 01, 01, 02, 30). Remember that JS months are zero-based.
		}{
			The date formatted on YYYY-MM-DD HH:MM:SS format. In the case above:
			2011-04-01 01:02:30
		}

	\section{Integration testing}
	Manual testing of the application as a whole.

	\subsection{Post a new thread}
		\ptestcase{
			The server is running and the browser is located at the forum start page.
		}{
			"Test user" is entered as login, and the login-button is clicked.
			A forum name is clicked. The "Create new thread"-button is clicked, a title and text is entered, and the post button is pressed. 
		}{
			A thread with the entered name is created and shows up immediately in the list of all threads. "Test user" is shown as the creator of the thread. The thread is shown when its title is clicked.
		}


	\subsection{Reply to a thread}
		\ptestcase{
			The server is running and the browser is located at a thread.
		}{
			The "Post"-button is pressed. A text field and a send-button shows up. A text is entered in the field and the button is pressed.
		}{
			The answer shows upp correctly and without reloading the page.
		}



\end{document}